Practical 1.
Сравнение двух сегментаторов.
В практическом задании я использовала два сегментатора, которые были предложены в задании: pragmatic segementer(Ruby) и NLTK sent tokenization(библиотека Python).
В качестве данных был использован текст на русском языке, а также на английском . 
Текст был предоставлен сегментаторам рандомно. 
Pragmatic segmenter был выполнен по инструкции, которая была в задани.В итоге pragmatic_segmenter на русском тексте сработал немного хуже, чем на английском тексте. В основном ошибки были в аббревиатурах и скоращениях.
А NLTK sent_tokenizer показал себя одинаково и на русском тексте и на английском. Но все же в каких-то местах текст на английском языке был токенизирован лучше. 

Примеры: 
'В собрании Лейденского музея сохранился древнеегипетский папирус «Речение Ипувера» (ок.',
 'XIII - XVIII вв.',
 'до н.э.)',
 'красочно описывающий трагические события смуты, сопровождавшейся разорением и распадом страны.',
 'По мнению ряда историков, речь в этом документе может идти о первом известном в истории массовом социальном движении, или даже «гражданской войне».',
 'Такого же характера другое древнеегипетское произведение «Пророчество Неферти» (ок.',
 'XV в. до н.э), где сказано:«Я показываю тебе сына в виде врага, брата в виде противника; Человек будет убивать своего отца… Будет страна мала, а её руководители многочисленны».'
 
 'The novel is the first semi-fictional work written by Ottaviani; previously, he had taken no creative license with the characters he depicted, portraying them strictly according to historical sources.',
 'Bone Sharps follows two scientists, Othniel Charles Marsh and Edward Drinker Cope (pictured), as they pursue their hotheaded and sometimes illegal acquisitions of fossils.',
 'Along the way, they encounter P. T. Barnum, Buffalo Bill, Alexander Graham Bell, Ulysses S. Grant, and other figures of the Gilded Age.'