\documentclass{article}
\usepackage[utf8]{inputenc}

\title{Project workflow comments
}
\author{Nikolay Babakov
 }
\date{January 2019}

\begin{document}

\maketitle

\section{Data collection}
A
1) Manually copy pasted easy fairy tales from https://iltasatu.org/
2) Manually copy pasted easy texts from https://revita.cs.helsinki.fi/
3) Parsed text with python from https://yle.fi/uutiset/osasto/selkouutiset/tosi_helppo/

B
1) Parsed text with python from https://yle.fi/uutiset/osasto/selkouutiset/
2) Parsed text with python from https://www.suomesta.ru/suomen_kiel/adaptirovannye-teksty-na-finskom/

C
1) Manually copy pasted easy texts from https://revita.cs.helsinki.fi/
2) Manually copy pasted translation of russian books with known difficulty from https://manybooks.net/

\section{Similar works analysis}
We looked through some related works about English anguage and about the languages with same structure such as Korean, Japanese and Turkish

Here is a short list of what the works we read
http://www.aclweb.org/anthology/W16-0531
http://www.aclweb.org/anthology/C88-2135
https://pdfs.semanticscholar.org/1324/80de063eeb5b39c1ed2f240b11dbfdcf455d.pdf
https://www.researchgate.net/publication/228891771_Toward_a_readability_index_for_Japanese_learners_of_EFL
https://www.researchgate.net/publication/323142424_Distributed_Readability_Analysis_Of_Turkish_Elementary_School_Textbooks

But the most important thing we have found was https://github.com/UniversalDependencies/UD_Finnish-TDT
We have installed it according to these instructions http://wiki.apertium.org/wiki/UDPipe

\section{Text processing, feature generation}

I divided all collected texts up to around ten setnences files and run UDPipe with them using this terminal command
for f in ./*.txt;do cat $f | udpipe --tokenize --tag --parse nob.udpipe
 > $f.parsed;done
 
 
 
 \section{Model training}
SVM 
 