\documentclass[dvipsnames, 10pt, compress]{beamer}

\usetheme{metropolis}
\usepackage{appendixnumberbeamer}

\usepackage{tikz-dependency}
\usepackage{caption}
%\usepackage{booktabs}
%\usepackage{tabularx}
\usepackage{alltt}
\usepackage[scale=2]{ccicons}

\usepackage{tikz-qtree}
\usepackage[linguistics]{forest}
\usepackage{pgfplots}
\usepgfplotslibrary{dateplot}

\usepackage{xspace}
\usepgflibrary{shapes.arrows}
\newcommand{\themename}{\textbf{\textsc{metropolis}}\xspace}


% commands from the paper
\newfontfamily\gtfont[Scale=1.1,Letters=SmallCaps]{Linux Libertine O}
\newcommand{\udtag}[1]{{\ll \textsc{#1}}}
\newcommand{\gtlabel}[1]{{\gtfont #1}}
\newcommand{\udlabel}[1]{{\tt #1}}
\newfontfamily\udfont[Scale=0.9,Letters=SmallCaps]{Linux Libertine O}
\newcommand{\utag}[1]{{\udfont#1}}
\newcommand{\ufeat}[1]{{\udfont#1}}
\newcommand{\tgl}[1]{{\em #1}}
\setmonofont[Scale=MatchLowercase]{DejaVu Sans Mono}

% commands from the paper

\newcommand*{\mystrut}{\rule[0.2\baselineskip]{0pt}{0.2\baselineskip}}
\newcommand{\redbox}[1]{\fcolorbox{red}{white}{\mystrut #1}}
\newcommand{\bluebox}[1]{\fcolorbox{ProcessBlue}{white}{\mystrut #1}}
\newcommand{\greenbox}[1]{\fcolorbox{YellowGreen}{white}{\mystrut #1}}
\newcommand{\redfillbox}[1]{\colorbox{red}{\textcolor{white}{{\bf #1}}}}
\newcommand{\bluefillbox}[1]{\colorbox{ProcessBlue}{\textcolor{white}{{\bf #1}}}}
\newcommand{\greenfillbox}[1]{\colorbox{YellowGreen}{\textcolor{white}{{\bf #1}}}}
%\newcommand*{\mybox}[1]{\framebox{\strut #1}}

\newcommand{\myarrow}[1][-45]{%
  \mathrel{%
    \text{$
     \begin{tikzpicture}[baseline = -0.5ex]
       \node[inner sep=0pt,outer sep=0pt,rotate = #1] (a) at (0,0)  {$\xrightarrow{}$};
    \end{tikzpicture}
    $}%
  }%
}%




\title{Class 11: Anaphora and co-reference resolution}
\date{}
\begin{document}
\tikzstyle{side arrow} = [draw=black!75, very thick, double arrow, minimum height=5.0cm, shape border rotate =#1, fill=gray!10]

\maketitle

\begin{frame}{Things and naming things}



\end{frame}

\begin{frame}{What is co-reference resolution?}

\begin{onlyenv}<1>
THE prime minister has fired secretary of state Priti Patel while telling 
her she wishes with all her heart it was the other way around. 

May confessed to Patel that although having secret meetings with the Israelis 
was a sackable offence, it paled in comparison to her own dire 
performance but “sadly nobody’s willing to pull the trigger.” 

Patel said: “It was so awkward. She said ‘You don’t know how often I’ve dreamt of 
sitting on your side of the desk, finally being summarily dismissed for 
my gross incompetence.’ 
\end{onlyenv}

\begin{onlyenv}<2>
\begin{center}
\includegraphics[width=0.4\textwidth]{graphics/theresa-may.jpg}
\includegraphics[width=0.4\textwidth]{graphics/priti-patel.jpg}
\end{center}
THE prime minister has fired secretary of state Priti Patel while telling 
her she wishes with all her heart it was the other way around. 

May confessed to Patel that although having secret meetings with the Israelis 
was a sackable offence, it paled in comparison to her own dire 
performance but “sadly nobody’s willing to pull the trigger.” 

Patel said: “It was so awkward. She said ‘You don’t know how often I’ve dreamt of 
sitting on your side of the desk, finally being summarily dismissed for 
my gross incompetence.’ 
\end{onlyenv}


\begin{onlyenv}<3>
\begin{center}
\includegraphics[width=0.4\textwidth]{graphics/theresa-may.jpg}
\includegraphics[width=0.4\textwidth]{graphics/priti-patel.jpg}
\end{center}
\colorbox{ProcessBlue}{\textcolor{white}{{\bf THE prime minister}}} has fired secretary of state Priti Patel while telling 
her \colorbox{ProcessBlue}{\textcolor{white}{{\bf she}}} wishes with all \bluebox{her} heart it was the other way around. 

\colorbox{ProcessBlue}{\textcolor{white}{{\bf May}}} confessed to Patel that although having secret meetings with the Israelis 
was a sackable offence, it paled in comparison to \bluebox{her} own dire 
performance but “sadly nobody’s willing to pull the trigger.” 

Patel said: “It was so awkward. \colorbox{ProcessBlue}{\textcolor{white}{{\bf She}}} said ‘You don’t know how often \colorbox{ProcessBlue}{\textcolor{white}{{\bf I}}}’ve dreamt of 
sitting on your side of the desk, finally being summarily dismissed for 
\bluebox{my} gross incompetence.’ 
\end{onlyenv}

\begin{onlyenv}<4>
\begin{center}
\includegraphics[width=0.4\textwidth]{graphics/theresa-may.jpg}
\includegraphics[width=0.4\textwidth]{graphics/priti-patel.jpg}
\end{center}
THE prime minister has fired \colorbox{red}{\textcolor{white}{{\bf secretary of state}}} \colorbox{red}{\textcolor{white}{{\bf Priti Patel}}} while telling 
\colorbox{red}{\textcolor{white}{{\bf her}}} she wishes with all her heart it was the other way around. 

May confessed to \colorbox{red}{\textcolor{white}{{\bf Patel}}} that although having secret meetings with the Israelis 
was a sackable offence, it paled in comparison to her own dire 
performance but “sadly nobody’s willing to pull the trigger.” 

\redfillbox{Patel} said: “It was so awkward. She said ‘\colorbox{red}{\textcolor{white}{{\bf You}}} don’t know how often I’ve dreamt of 
sitting on \redbox{your} side of the desk, finally being summarily dismissed for 
my gross incompetence.’ 
\end{onlyenv}

\begin{onlyenv}<5>
\begin{center}
\includegraphics[width=0.4\textwidth]{graphics/theresa-may.jpg}
\includegraphics[width=0.4\textwidth]{graphics/priti-patel.jpg}
\end{center}
\bluefillbox{THE prime minister} has fired \colorbox{red}{\textcolor{white}{{\bf secretary of state}}} \colorbox{red}{\textcolor{white}{{\bf Priti Patel}}} while telling 
\colorbox{red}{\textcolor{white}{{\bf her}}} \bluefillbox{she} wishes with all \bluebox{her} heart it was the other way around. 

\bluefillbox{May} confessed to \colorbox{red}{\textcolor{white}{{\bf Patel}}} that although having secret meetings with the Israelis 
was a sackable offence, it paled in comparison to \bluebox{her} own dire 
performance but “sadly nobody’s willing to pull the trigger.” 

\redfillbox{Patel} said: “It was so awkward. \bluefillbox{She} said ‘\colorbox{red}{\textcolor{white}{{\bf You}}} don’t know how often \bluefillbox{I}’ve dreamt of 
sitting on \redbox{your} side of the desk, finally being summarily dismissed for 
\bluebox{my} gross incompetence.’ 
\end{onlyenv}


\end{frame}

\begin{frame}{Noun phrases and reference}

\fbox{\parbox{\textwidth}{
\begin{itemize}
\item NPs usually refer to entities in the world
\item NPs may co-refer, meaning they refer to the same entity
\item They may also be nested
\end{itemize}}}

Однажды \bluefillbox{Пушкин} написал письмо \redfillbox{Рабиндранату Тагору}. 
«\redfillbox{Дорогой далекий друг}, — писал \bluefillbox{он}, — \bluefillbox{я} \redfillbox{Вас} не знаю, и \redfillbox{Вы} \bluefillbox{меня} не знаете.
Очень хотелось бы познакомиться. Всего хорошего. \bluefillbox{Саша}».
Когда письмо принесли, \redfillbox{Тагор} предавался самосозерцанию. 
Так погрузился, хоть режь \redfillbox{его}. 
\greenfillbox{\redbox{\textcolor{black}{Его}} жена} толкала, толкала, письмо подсовывала — не видит. 
\redfillbox{Он}, правда, по-русски читать не умел. Так и не познакомились.



\end{frame}

\begin{frame}{Kinds of reference}

\begin{columns}

\column{0.2\textwidth}

\begin{tikzpicture}[overlay]
\node[draw] at (1,3.0) {Easier};
\node[draw] at (1,-3.0) {Harder};
\node at (1,0) [side arrow=90] {\rotatebox{90}{~}};
\end{tikzpicture}


\column{0.8\textwidth}

Interesting linguistics 
~\\

\begin{tabular}{ll}

\textbf{Bound variables} & She hurt \emph{herself} \\
                         & Я имею \emph{свой} баян \\
% Bound variables [She hurt herself]

%% Only possible analysis is coreferent with the subject

% Free variables [She saw her pay increase] -- [Она видела ее/свой]
~ & ~ \\
\textbf{Free variables} & Maša read \emph{her} book \\
                        & \emph{Она} очень нравилась. \\

%% Might be coreferent with the subject or not

% Referring expressions 
~ & ~ \\

\textbf{Referring expressions} & Carles Puigdemont \\
                               & Catalan president \\
                               & Puigdemont \\ 
                               & president of Catalonia \\
                               & President Puigdemont \\
%% 
\end{tabular}

More frequent

\end{columns}

\end{frame}

\begin{frame}{Coreference, anaphora, cataphora}

\begin{itemize}
  \item  \textbf{Coreference} 
  \begin{itemize}  
    \item Two \emph{mentions} (NPs) refer to the same entity 
    \item May be identical or completely different
  \end{itemize}
  \item \textbf{Anaphora, Cataphora} 
  \begin{itemize}
    \item Interpretation is in some way dependent on an antecedent
    \item Traditionally the antecedent came first, but not always the case.
  \end{itemize}
\end{itemize}

\end{frame}


\begin{frame}{Cataphora}

%From the corner of the divan of Persian saddlebags on which he was lying, smoking, as was his custom, innumerable cigarettes, Lord Henry Wotton could just catch the gleam of the honey-sweet and honey-coloured blossoms of a laburnum, whose tremulous branches seemed hardly able to bear the burden of a beauty so flame-like as theirs;...

% С покрытого персидскими чепраками дивана, на котором лежал лорд Генри Уоттон, куря, как всегда, одну за другой бесчисленные папиросы, был виден только куст ракитника -- его золотые и душистые, как мед, цветы жарко пылали на солнце, а трепещущие ветви, казалось, едва выдерживали тяжесть этого сверкающего великолепия \ldots


(Oscar Wilde -- The Picture of Dorian Grey)
%
%<padre_angolano> spectie: "Еще раз сверившись с бумажкой, на которой он все-таки вопреки советам Васи записал адрес, Пафнутьев направился к стоянке такси"
%<padre_angolano> spectie: "Внимательно вглядываясь в несущуюся под ногами тропинку, на которую он обычно садился, Митя широко раскрыл крылья, повернул их навстречу бьющему в лицо воздуху"

\end{frame}

\begin{frame}{Anaphora vs. coreference}

We went to a concert last night. The tickets were really expensive

\begin{itemize}
  \item Not all anaphoric relations are coreferential, e.g. bridging anaphora
  \item Multiple identical NP matches are often coreferential but not anaphoric
\end{itemize}

\end{frame}

% example

\begin{frame}{Two different things}


\begin{tikzpicture}[overlay]
\node at (7,3.0) {\textbf{Anaphora resolution}};
\node at (1,2.0) {\textbf{Text}};
\node at (1,0.5) {\textbf{World}};
\node[draw] at (5,2.0) {~};
\node[draw] at (9,2.0) {~};
\node[draw] at (5,0.5) {~};
%\node[draw=black!75, thin, double arrow, minimum height=3.0cm, fill=gray!10] at (7,2.0) {~~~~} ;
%\node[draw=black!75, thin, double arrow, shape border rotate=90, minimum height=1.0cm, fill=gray!10] at (5,1.0) {~~~~} ;
\draw[<->,line width=1pt] (5.5,2.0) to (8.5,2.0);
\draw[<->,line width=1pt] (5.0,1.6) to (5.0,0.8);
\end{tikzpicture}

~\\

\begin{tikzpicture}[overlay]
\node at (7,0.5) {\textbf{Co-reference resolution}};
\node at (1,-0.5) {\textbf{Text}};
\node at (1,-2.0) {\textbf{World}};
\node[draw] at (5,-0.5) {~};
\node[draw] at (9,-0.5) {~};
\node[draw] at (7,-2.0) {~};
\draw[<->,line width=1pt] (5.5,-0.75) to (6.8,-1.8);
\draw[<->,line width=1pt] (8.5,-0.75) to (7.2,-1.8);
\end{tikzpicture}

\end{frame}

\begin{frame}{Applications}

Machine translation:

Text summarisation:

Information extraction:


\end{frame}

% ALGORITHMS

\section{Pronominal anaphora resolution}

\begin{frame}


\end{frame}

\begin{frame}{Hobbs' (1978) algorithm/1}

\begin{itemize}
  \item Simple syntax-based algorithm for 3rd person anaphoric pronouns
  \item Requires:
  \begin{itemize}
     \item Constituency parser
     \item Gender and number `checker'
     \begin{itemize}
        \item Parsers for English rarely include gender information for nouns
     \end{itemize}
  \end{itemize}
  \item Searches current and preceding sentences in a breadth-first, left-to-right
     manner, stops when it finds a matching NP
\end{itemize}

\end{frame}

\begin{frame}{Hobbs' (1978) algorithm/2}

\begin{itemize}
  \item Right to left search in current sentence 
  \item If not valid antecedent fine, try previous sentence
  \begin{itemize}
    \item Left to right breadth-first search
  \end{itemize}
 
\end{itemize}

\end{frame}

\begin{frame}{Hobbs' (1978) algorithm/3}

\begin{onlyenv}<1>
\includegraphics[width=\textwidth]{graphics/drone-training-2x.png}
\end{onlyenv}
\begin{onlyenv}<2>
\includegraphics[width=\textwidth]{graphics/drone-training-2x-2.png}
\end{onlyenv}


\end{frame}

\begin{frame}{Hobbs' (1978) algorithm/4}
%My drone keeps flying into the wrong rooms.
%Do you have anything to discourage it?

\begin{columns}

\column{0.6\textwidth}
%\begin{tikzpicture}
%\Tree [.S [.DP My [.NP drone]] [.VP [.V keeps] [.Comp flying [.PP into [.DP the [.NP [.AP [.A wrong] [.N rooms]]]]]]]]
%\end{tikzpicture}
\scalebox{0.8}{
\begin{forest}
[S$_1$ [NP$_1$ [My drone]] [VP [V [keeps flying]] [PP [into] [NP$_2$ [the wrong rooms]]]]]
%[S 
%  [NP [My drone] ]
%  [VP [keeps 
%        flying [PP [into] [the [NP [wrong] [rooms]]]]]]
%]
\end{forest}
}
~\\
~\\
\begin{itemize}
   \item Start search in NP$_5$ in S$_2$
   \item Reject NP$_4$, no intervening NP
   \item Reject NP$_3$, feature mismatch
   \item Move to S$_1$
   \item Accept NP$_1$
\end{itemize}
\column{0.4\textwidth}
\scalebox{0.8}{
\begin{forest}
[S$_2$ [Do] [S$_3$ [NP$_3$ [You]] [VP [have] [SC [NP$_4$ [anything]] [Inf [to] [VP [discourage] [NP$_5$ [it]]]]]]]]
\end{forest}
}
\end{columns}

\end{frame}

\begin{frame}{Log-linear model/1} % General idea

\begin{itemize}
  \item Supervised machine learning approach
  \item Requires corpus where each pronoun has been linked with its antecedent
\end{itemize}

\begin{itemize}
  \item Extract positive and negative examples
  \item Train binary classifier 
  \begin{itemize}
     \item True: is co-referent
     \item False: is not co-referent
  \end{itemize}
\end{itemize}

\end{frame}

\begin{frame}{Log-linear model/2} % Positive and negative examples
%My drone keeps flying into the wrong rooms.
%Do you have anything to discourage it?

Let's take our previous example:

Positive example:
\begin{itemize}
 \item (it, my drone)
\end{itemize}

Negative examples:
\begin{itemize}
 \item (it, anything)
 \item (it, you)
 \item (it, the wrong rooms)
\end{itemize}




\end{frame}

\begin{frame}{Log-linear model/3} % Features

\begin{itemize}
  \item \textbf{strict gender} [true, false]
  \item \textbf{compatible gender} [true, false]
  \item \textbf{strict number} [true, false]
  \item \textbf{compatible number} [true, false]
  \item \textbf{sentence distance} [0, 1, 2, \ldots]
  \item \textbf{Hobbs distance} [0, 1, 2, \ldots] 
  \item \textbf{grammatical role} [subject, object, \ldots] 
  \item \textbf{linguistic form} [proper, def, indef, pronoun]
\end{itemize}

Can you think of some other useful features ? 

\end{frame}

\begin{frame}{Log-linear model/4} % Applying the model

\begin{itemize}
  \item For each pronoun,
  \begin{itemize}
     \item For each NP we have seen so far,
     \begin{itemize}
       \item Classify if NP is an antecedent of the pronoun
     \end{itemize}
  \end{itemize}
\end{itemize}


\end{frame}

\section{Co-reference resolution}


\begin{frame}


\end{frame}


\begin{frame}{Co-reference chains}

Однажды \bluefillbox{Пушкин} написал письмо \redfillbox{Рабиндранату Тагору}. 
«\redfillbox{Дорогой далекий друг}, — писал \bluefillbox{он}, — \bluefillbox{я} \redfillbox{Вас} не знаю, и \redfillbox{Вы} \bluefillbox{меня} не знаете.
Очень хотелось бы познакомиться. Всего хорошего. \bluefillbox{Саша}».
Когда письмо принесли, \redfillbox{Тагор} предавался самосозерцанию. 
Так погрузился, хоть режь \redfillbox{его}. 
\greenfillbox{Его жена} толкала, толкала, письмо подсовывала — не видит. 
\redfillbox{Он}, правда, по-русски читать не умел. Так и не познакомились.

\begin{tabular}{l}
{ Пушкин, он$_1$, я$_1$, меня$_1$ } \\ 
{ Рабиндранату Тагору, Дорогой далекий друг, Вас$_1$, Вы$_1$, Тагор, его$_1$, Он} \\
\end{tabular}

\end{frame}

\begin{frame}{General algorithm}

\end{frame}


\begin{frame}{Additional features}

\end{frame}


\begin{frame}{Rule-based models} % xrenner

Input is a dependency tree.

Constraint-based, rules like:

$C = <ANA,  ANT,  DIR, DIST, PROP>$

\begin{itemize}
  \item ANA, ANT = constraints on the anaphor and antecedent
  \item DIR = direction (e.g. forward, backwards)
  \item DIST = how far to look (in sentences)
  \item PROP = should features (e.g. gender) be propagated?
\end{itemize}

\end{frame}

\section{Evaluation}

% EVALUATION

\begin{frame}{Model-Theoretic coreference scoring}


\end{frame}

\begin{frame}{Other metrics}


\end{frame}

\begin{frame}{Tools and resources}

% ruCoRef http://rucoref.maimbava.net/

\end{frame}

\begin{frame}{xrenner}

\end{frame}

\begin{frame}{Stanford CoreNLP}

\end{frame}


\section{Shared tasks}
% SHARED TASK(S)

\begin{frame}{RuEval-2014}

\end{frame}

\section{Practical}

% PRACTICAL



\end{document}
